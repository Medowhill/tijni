\documentclass[acmsmall,screen]{acmart}

\AtBeginDocument{%
  \providecommand\BibTeX{{%
    \normalfont B\kern-0.5em{\scshape i\kern-0.25em b}\kern-0.8em\TeX}}}

\bibliographystyle{ACM-Reference-Format}
\citestyle{acmauthoryear}   %% For author/year citations

\usepackage{mathtools}
\usepackage{amsfonts}
\usepackage{amssymb}
\usepackage{mathpartir}
\usepackage{verbatim}
\usepackage{hyperref}
\usepackage{xspace}
\usepackage{stmaryrd}
\usepackage{multicol}
\usepackage{listings}

\lstdefinestyle{cstyle}{
  frame=tb,
  aboveskip=3mm,
  belowskip=3mm,
  showstringspaces=false,
  columns=fixed,
  basicstyle={\small\ttfamily},
  numbers=none,
  numberstyle=\tiny\color{gray},
  keywordstyle=\color{blue},
  commentstyle=\color{dkgreen},
  stringstyle=\color{purple},
  frame=single,
  breaklines=true,
  breakatwhitespace=true,
  tabsize=3,
  language=C,
}
\def\inline{\lstinline[style=javastyle]}

\lstdefinestyle{javastyle}{
  frame=tb,
  aboveskip=3mm,
  belowskip=3mm,
  showstringspaces=false,
  columns=fixed,
  basicstyle={\small\ttfamily},
  numbers=none,
  numberstyle=\tiny\color{gray},
  keywordstyle=\color{blue},
  commentstyle=\color{dkgreen},
  stringstyle=\color{purple},
  frame=single,
  breaklines=true,
  breakatwhitespace=true,
  tabsize=3,
  language=Java,
}
\def\inline{\lstinline[style=javastyle]}

\begin{document}

\title{Detecting Memory Bugs in JNI Programs}
\author{Jaemin Hong}
\affiliation{
  \institution{KAIST}
}
\maketitle

\section{Background}

The Java Native Interface (JNI) allows Java programs to call functions written
in C or C++. Functions written in C or C++ are usually called native methods
in JNI programming. The following is a code snippet from an example JNI program
(some unimportant parts are omitted):

\begin{lstlisting}[style=javastyle]
class Hello {
    native void hello();
    public static void main(String[] args) { (new Hello()).hello(); }
}
\end{lstlisting}
\vspace*{-1em}
\begin{flushright}
{\footnotesize\tt Hello.java}
\end{flushright}
\vspace*{-0.5em}
\begin{lstlisting}[style=cstyle]
JNIEXPORT void JNICALL Java_Hello_hello(JNIEnv *env) {
    printf("Hello world!\n");
}
\end{lstlisting}
\vspace*{-1em}
\begin{flushright}
{\footnotesize\tt Hello.c}
\end{flushright}

The {\tt hello} method of the {\tt Hello} class is defined as a native method.
When the method is invoked, the Java Virtual Machine (JVM) calls the
\verb!Java_Hello_hello! function, which is defined in the {\tt Hello.c} file.
Therefore, the program prints \verb!"Hello world\n"! to the standard output.

Programmers use the JNI to write compute-intensive parts of their programs or to
reuse code written in C or C++ in Java programs without implementing the whole
logic again. Lots of programs using the JNI exist in the real world. For
example, the {\tt java.util.zip} package of the \citet{javautilzips}
uses the JNI for high performance.

One typical pattern to use the JNI is the peer pattern, which was introduced by
\citet{liang1999java}. In this pattern, a pointer to a dynamically allocated
memory block should be stored in a Java object. For instance, consider the
following code (some unimportant parts are omitted):

\begin{lstlisting}[style=javastyle]
class Peer {
    native long malloc(int size);
    native void free(long ptr);
    long ptr;
    Peer(int size) { ptr = malloc(size); }
    void destroy() { free(ptr); }
}
\end{lstlisting}
\vspace*{-1em}
\begin{flushright}
{\footnotesize\tt Peer.java}
\end{flushright}
\vspace*{-0.5em}
\begin{lstlisting}[style=cstyle]
JNIEXPORT jlong JNICALL Java_Peer_malloc(JNIEnv *env, jint size) {
    return (jlong) malloc(size);
}
JNIEXPORT void JNICALL Java_Peer_free(JNIEnv *env, jlong ptr) {
    free((void *) ptr);
}
\end{lstlisting}
\vspace*{-1em}
\begin{flushright}
{\footnotesize\tt Peer.c}
\end{flushright}

Since pointers are not Java values, they are represented as 64-bit integers in
Java objects. Note that {\tt jlong} and {\tt jint} are types respectively denoting
{\tt long} and {\tt int} of Java in native code. The pattern is useful to make
Java wrapper classes of native structs and classes.

\section{Motivation}

The peer pattern can be the source of a memory bug in practice. If a pointer is
used after being freed, then the JVM crashes immediately due to access to a
dangling pointer. On the other hand, if a Java object carrying a pointer is
garbage-collected before the pointer is freed, then the memory allocated by the
native side will never be reclaimed. It is a memory leak.

Although these kinds of memory bugs are common in C or C++ programming, I
believe that they are more harmful in JNI programming than C or C++
programming. The reasons follow:
\begin{itemize}
    \item Java programmers are unfamiliar with memory bugs since the JVM
      manages memory in Java programs. Besides, they used to debug
      their programs with logs provided by the JVM, but the JVM shows less useful
      messages when it crashes while executing native code.
    \item In many cases, programmers use existing native libraries for their
      programs. They implement only a small portion of the native code,
      which is required for interoperation. Thus, when they
      find memory bugs, they need to inspect the native code that they did not
      write.
    \item Since memory allocation and deallocation happen in the native side,
      programmers cannot easily determine whether invoking a certain Java method
      will make memory blocks be allocated or deallocated.
\end{itemize}

Programmers have suffered from memory bugs while using the JNI. For instance,
\citet{libgdx} is a Java game development framework using the JNI. Although the
framework itself has been tested a lot, game developers have trouble with using
the framework correctly. \citet{libgdxissue} claimed that he had experienced
crashes in the issue tracker of the libGDX GitHub
repository. However, the reason of the crashes was found in his code: {\tt
dispose} was called twice, and it caused double-free, which made the JVM crash.
Several similar issues are in the issue tracker.

In this project, I will propose a novel static analyzer to detect memory bugs
that can be found in the peer pattern.
There has been multiple approaches to detect memory bugs in JNI programs
~\cite{tan2006safe,kondoh2008finding,lee2010jinn,hong2017museum}. However, most
of them aimed different sorts of memory bugs in the JNI or used dynamic
analyses. To the best of my knowledge, there is no static analyzer
detecting memory bugs due to pointers stored in Java objects.

\section{My Approach}

The most challenging part of the project is that the analyzer needs to deal
with programs written in more than one language. It must analyze code written in
Java and C (or C++). The analyzer will analyze native code first and then Java
code. More precisely, the first part of the analyzer will extract the summaries of
functions in native code. The summaries need to contain information about functions
that allocate or deallocate memory. The second part of the
analyzer will analyze Java code with the summaries. Due to the summaries, the
analyzer can determine whether a particular Java method will allocate or
deallocate memory. In this way, the analyzer will be able to detect memory bugs,
including dangling pointers and memory leaks, in JNI programs.

\section{Expected Results}

I expect that the analyzer will be able to detect memory bugs in small JNI programs.
However, to reduce false alarms, the anazlyer needs to use proper sensitivites.
At the same time, to find memory bugs in large-scale JNI programs,
which are typical cases in the real world, the analyzer needs to use proper
advanced techniques, for example sparse analyses.

\bibliography{ref}

\end{document}
